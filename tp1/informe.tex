\documentclass{article}
\usepackage[utf8]{inputenc}
\usepackage{listings}
\usepackage{amsmath,amssymb}
\usepackage{graphicx}
\usepackage{xcolor}

\definecolor{codegreen}{rgb}{0,0.6,0}
\definecolor{codegray}{rgb}{0.5,0.5,0.5}
\definecolor{codepurple}{rgb}{0.58,0,0.82}
\definecolor{backcolour}{rgb}{0.95,0.95,0.92}

\lstdefinestyle{mystyle}{
    backgroundcolor=\color{backcolour},   
    commentstyle=\color{codegreen},
    keywordstyle=\color{magenta},
    numberstyle=\tiny\color{codegray},
    basicstyle=\ttfamily\footnotesize,
    breakatwhitespace=false,         
    breaklines=true,                 
    captionpos=b,                    
    keepspaces=true,                 
    numbers=left,                    
    numbersep=5pt,                  
    showspaces=false,                
    showstringspaces=false,
    showtabs=false,                  
    tabsize=2
}

\lstset{style=mystyle}


\title{	\ Trabajo Práctico Nro 1: Algoritmos Greedy y Dividisión y conquista}

\author{    Nestor Huallpa, \textit{Padrón Nro. 88614}\\
            \texttt{ huallpa.nestor@gmail.com }\\\\  
            nnnnnnnnnnnnnnnnnnnnn, \textit{Padrón Nro. xxxx}\\
            \texttt{ nnn@yahoo.com.ar }\\\\              
            \texttt{\footnotesize 1º Entrega: xx/xx/xxx}\\
            \\\\\\\\\\\\\\\\\\
            \normalsize{1do. Cuatrimestre de 2020}\\ 
            \normalsize{75.29/95.06 Teoría de Algoritmos I} \\
            \normalsize{Facultad de Ingeniería, Universidad de Buenos Aires} \\}
       
\date{}

\begin{document}

\maketitle
% quita el número en la primer página
\thispagestyle{empty}

\newpage{}
\tableofcontents

% quita el número en la primer página
\thispagestyle{empty}

\newpage{}

\newpage
\section{Introducción}

En el presente trabajo plantearemos dos soluciónes mediante algoritmos para al problema de ausentismo de una empresa y sobre una nueva regulación industrial. 

\section{Un problema de ausentismo}

\subsection{Descripción del problema}

Una empresa de tercerización laboral nos convoca para que le ayudemos con un problema de ausentismo laboral. 
Tiene un conjunto de \(n\) empleados que realizan tareas en diferentes puntos de la ciudad. 
El turno de cada empleado \(i\) comienza en \(T_i(i)\) y termina en \(T_f(i)\) y durante todo ese lapso tiene que estar en la ubicación establecida. 
La dirección de la empresa sospecha que algunos de sus empleados suelen faltar sin aviso. Para verificarlo contrataron a la empresa “Dystopian Technologies Inc.” (DTI). 
Esta empresa implanta un microchip con un código único en cada empleado. Mediante rastreo satelital pueden conocer dónde se encuentra cada chip implantado en cualquier momento. Además posee el cronograma completo de las tareas.

DTI brinda un sistema que mediante una consulta (encendido / apagado) nos devolverá cuáles empleados aún no controlados y en horario de trabajo se encuentran en su sitio y cuáles no.
\subsection{Hipótesis}
\begin{itemize}
    \item Los tiempos informados son enteros de 0 en adelante.
    \item DTI les cobra por cada encendido / apagado.
    \item Cada encendido / apagado es casi instantáneo y se lo programa para algún valor de t entero.
    \item Cada encendido / apagado (y su consecuente rastreo) es \(O(1)\).
    \item El empleado una vez en su puesto no se retira hasta concluir su turno.
\end{itemize}
\subsection{Descripción del algoritmo}

\subsection{Pseudocódigo del algoritmo}

\subsection{Análisis del algotimo}

\section{Una nueva regulación industrial}
\section{Conclusión}




\end{document}